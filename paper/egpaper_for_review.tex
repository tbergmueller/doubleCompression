\documentclass[10pt,twocolumn,letterpaper]{article}

\usepackage{icb}
\usepackage{times}
\usepackage{epsfig}
\usepackage{graphicx}
\usepackage{amsmath}
\usepackage{amssymb}

% Include other packages here, before hyperref.

% If you comment hyperref and then uncomment it, you should delete
% egpaper.aux before re-running latex.  (Or just hit 'q' on the first latex
% run, let it finish, and you should be clear).
%\usepackage[pagebackref=true,breaklinks=true,letterpaper=true,colorlinks,bookmarks=false]{hyperref}

%\icbfinalcopy % *** Uncomment this line for the final submission

\def\icbPaperID{****} % *** Enter the IJCB Paper ID here
\def\httilde{\mbox{\tt\raisebox{-.5ex}{\symbol{126}}}}

% Pages are numbered in submission mode, and unnumbered in camera-ready
\ificbfinal\pagestyle{empty}\fi
\begin{document}

%%%%%%%%% TITLE
\title{\LaTeX\ Author Guidelines for ICB 2015 Proceedings [Based on CVPR]}

\author{First Author\\
Institution1\\
Institution1 address\\
{\tt\small firstauthor@i1.org}
% For a paper whose authors are all at the same institution,
% omit the following lines up until the closing ``}''.
% Additional authors and addresses can be added with ``\and'',
% just like the second author.
% To save space, use either the email address or home page, not both
\and
Second Author\\
Institution2\\
First line of institution2 address\\
{\tt\small secondauthor@i2.org}
}

\maketitle
\thispagestyle{empty}

%%%%%%%%% ABSTRACT
\begin{abstract}
   The ABSTRACT is to be in fully-justified italicized text, at the top
   of the left-hand column, below the author and affiliation
   information. Use the word ``Abstract'' as the title, in 12-point
   Times, boldface type, centered relative to the column, initially
   capitalized. The abstract is to be in 10-point, single-spaced type.
   Leave two blank lines after the Abstract, then begin the main text.
   Look at previous ICB abstracts to get a feel for style and length. 
\end{abstract}

%%%%%%%%% BODY TEXT
\section{Introduction}
Done by Mr. Uhl, including related work?

\section{Test data generation}
TODO: Thomas
\begin{itemize}
 \item Used Database is IITD, because Manual Groundtruth available, as in \cite{severeCompression}
 \item Describe the used methods and implementations
  \begin{itemize}
  \item JPG
  \item JXR
  \item JP2k
  \end{itemize}
 \item Describe the process of single- and double compression
 \item Per-Image quality
 \item Analyze Database and provide quality measures (accuracy, stddev, ...)
 \item Argue that this data base is sufficient
\end{itemize}






\section{Evaluation methods}
Introduction: Thomas


\subsection{Full-referenced quality metrics}
Todo Lefteris:
\begin{itemize}
 \item Which quality metrics were in the selection
 \item Which were chosen
 \item Why have you chosen these
 \item Give a very very brief introduction (rather referencing!) of what the quality metrics are about and the most characteristic features
 \item Results and findings of this evaluation
\end{itemize}

\subsection{Segmentation error rates}
TODO TB:
\begin{itemize}
 \item Calculate CR's in bpp
 \item Check whether our method is the same as E1 and/or E2 from this paper: \cite{severeCompression}
 \item If not, impl. E1, is E2 also necessary?
 \item Compare referenced to non-referenced
 \item Comparision
 \item Can we probably use their data to draw longer graphs
 \item Results and Findings of this evaluation
\end{itemize}


\subsection{Equal Error rate}
To assess the total impact on the System, the EER is computed
TODO TB

\begin{itemize}
 \item Brief introduction
 \item Results and findings of this evaluation
\end{itemize}


\section{Results}
\subsection{Schnoell-Correlation-method}
TODO: Martin: Introduce your method here and argue why it is better than spearman

\subsection{Correlation of Evaluation methods}
TODO: Martin:
Provide sensible correlation results and analyse


\section{Conclusion}
TODO: Martin Schnöll


{\small
\bibliographystyle{ieee}
\bibliography{egbib}
}

\end{document}
